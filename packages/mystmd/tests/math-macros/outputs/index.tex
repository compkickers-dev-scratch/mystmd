\documentclass{article}
\PassOptionsToPackage{short, nodayofweek}{datetime}

\input{curvenote.def}

%%%%%%%%%%%%%%%%%%%%%%%%%%%%%%%%%%%%%%%%%%%%%%%%%%
%%%%%%%%%%%%%%%%%  math commands  %%%%%%%%%%%%%%%%
\newcommand{\three}{d}
\newcommand{\one}{x}
\newcommand{\five}{x = \one}
\newcommand{\six}{d = \three}
\newcommand{\seven}{d = \six}
%%%%%%%%%%%%%%%%%%%%%%%%%%%%%%%%%%%%%%%%%%%%%%%%%%


% colors for hyperlinks
\hypersetup{colorlinks=true, allcolors=blue}
\hypersetup{
pdftitle={\@title},
pdfsubject={},
pdfauthor={\@author},
pdfkeywords={},
addtopdfcreator={Written in Curvenote}
}

\usepackage{curvenote}

\title{Testing Math Plugins}

\newdate{articleDate}{1}{1}{2024}
\date{\displaydate{articleDate}}

\author{}

\begin{document}

\maketitle
\keywords{}

\section{No plugins}

\begin{equation}
a^2 + b^2 = c^2
\end{equation}

\section{Simple plugin}

Project frontmatter should give us \texttt{d}

\begin{equation}
d = \three
\end{equation}

Page should override and we should see \texttt{x}

\begin{equation}
x = \one
\end{equation}

\section{Macros should recurse}

Page frontmatter should fill in this project macro

\begin{equation}
\five
\end{equation}

Project frontmatter should fill in this page macro

\begin{equation}
\six
\end{equation}

Double recurse

\begin{equation}
\seven
\end{equation}
\end{document}
